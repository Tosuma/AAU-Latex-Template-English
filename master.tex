% !TEX root = ./master.tex
%%%%%%%%%%%%%%%%%%%%%%%%%%%%%%%%%%%%%%%%%%%%%%%%
%      DO NOT TOUCH ANYTHING IN THIS FILE      %
%      UNLESS YOU KNOW WHAT YOU ARE DOING      %
%                                              %
%        New packages should be included       %
%             in the file 'packages'           %
%%%%%%%%%%%%%%%%%%%%%%%%%%%%%%%%%%%%%%%%%%%%%%%%


\documentclass[11pt,oneside,a4paper,openright]{report}
%%%%%%%%%%%%%%%%%%%%%%%%%%%%%%%%%%%%%%%%%%%%%%%%
% Language, Encoding and Fonts
% http://en.wikibooks.org/wiki/LaTeX/Internationalization
%%%%%%%%%%%%%%%%%%%%%%%%%%%%%%%%%%%%%%%%%%%%%%%%
% Select encoding of your inputs. Depends on
% your operating system and its default input
% encoding. Typically, you should use
%   Linux  : utf8 (most modern Linux distributions)
%            latin1 
%   Windows: ansinew
%            latin1 (works in most cases)
%   Mac    : applemac
% Notice that you can manually change the input
% encoding of your files by selecting "save as"
% an select the desired input encoding.
\usepackage[utf8]{inputenc}

% Make latex understand and use the typographic
% rules of the language used in the document.
%%% Change this to danish for danish typesettings %%%
%%% (automatically generated text)                %%%
\usepackage[english]{babel}

% Use the palatino font
\usepackage[sc]{mathpazo}
\linespread{1.05} % Palatino needs more leading (space between lines)

% Choose the font encoding
\usepackage[T1]{fontenc}


%%%%%%%%%%%%%%%%%%%%%%%%%%%%%%%%%%%%%%%%%%%%%%%%
% Page Layout
% http://en.wikibooks.org/wiki/LaTeX/Page_Layout
%%%%%%%%%%%%%%%%%%%%%%%%%%%%%%%%%%%%%%%%%%%%%%%%
% Change margins, papersize, etc of the document
\usepackage[
  inner=28mm, % Left margin on an odd page
  outer=41mm, % Right margin on an odd page
]{geometry}

% Modify how \chapter, \section, etc. look
% The titlesec package is very configureable
\usepackage{titlesec}
\titleformat{\chapter}[display]{\normalfont\huge\bfseries}{\chaptertitlename\ \thechapter}{20pt}{\Huge}
\titleformat*{\section}{\normalfont\Large\bfseries}
\titleformat*{\subsection}{\normalfont\large\bfseries}
\titleformat*{\subsubsection}{\normalfont\normalsize\bfseries}
%\titleformat*{\paragraph}{\normalfont\normalsize\bfseries}
%\titleformat*{\subparagraph}{\normalfont\normalsize\bfseries}

% Setting level of numbering (default for "report" is 3)
% With '-1' you have non number also for chapters
\setcounter{secnumdepth}{3}

% Clear empty pages between chapters
\let\origdoublepage\cleardoublepage
\newcommand{\clearemptydoublepage}{%
  \clearpage
  {\pagestyle{empty}\origdoublepage}%
}
\let\cleardoublepage\clearemptydoublepage

% Change the headers and footers
\usepackage{fancyhdr}
\pagestyle{fancy}
\fancyhf{} % Delete everything
\renewcommand{\headrulewidth}{0pt} % Remove the horizontal line in the header

%%%%%%%%%%%%%%%%%%
%%% Onepage setup - with fixed position for page number and text

\fancyhead[L]{\small\nouppercase\leftmark}
\fancyhead[R]{\thepage}

%%% Twopage setup (book-style) with alternating position
%%% for page number and text 
%%% !!REMEMBER!! To change margin settings in "master.tex"

% \fancyhead[RE]{\small\nouppercase\leftmark} %even page - chapter title
% \fancyhead[LO]{\small\nouppercase\rightmark} %uneven page - section title
% \fancyhead[LE,RO]{\thepage} %page number on all pages
%%%%%%%%%%%%%%%%%%

% Do not stretch the content of a page. Instead,
% insert white space at the bottom of the page
\raggedbottom

% Enable arithmetics with length. Useful when
% typesetting the layout.
\usepackage{calc}


%%%%%%%%%%%%%%%%%%%%%%%%%%%%%%%%%%%%%%%%%%%%%%%%
% Bibliography
% http://en.wikibooks.org/wiki/LaTeX/Bibliography_Management
%%%%%%%%%%%%%%%%%%%%%%%%%%%%%%%%%%%%%%%%%%%%%%%%

% Adds the bibliography used for citing
\usepackage[backend=biber,
  bibencoding=utf8,
  style=numeric-comp
]{biblatex}
\addbibresource{bib/mybib}


%%%%%%%%%%%%%%%%%%%%%%%%%%%%%%%%%%%%%%%%%%%%%%%%
% Misc
%%%%%%%%%%%%%%%%%%%%%%%%%%%%%%%%%%%%%%%%%%%%%%%%

% Add bibliography and index to the table of
% contents
\usepackage[nottoc]{tocbibind}

% Add the command \pageref{LastPage} which refers to the
% page number of the last page
\usepackage{lastpage}

% Add todo notes in the margin of the document
\usepackage[
  %  disable, %turn off todonotes
  colorinlistoftodos, %enable a coloured square in the list of todos
  textwidth=\marginparwidth, %set the width of the todonotes
  textsize=scriptsize, %size of the text in the todonotes
]{todonotes}

% Add quotes
\usepackage{csquotes}


%%%%%%%%%%%%%%%%%%%%%%%%%%%%%%%%%%%%%%%%%%%%%%%%
% Hyperlinks
% http://en.wikibooks.org/wiki/LaTeX/Hyperlinks
%%%%%%%%%%%%%%%%%%%%%%%%%%%%%%%%%%%%%%%%%%%%%%%%

% Enable hyperlinks and insert info into the pdf
% file. Hypperref should be loaded as one of the 
% last packages
\usepackage[pdfpagelabels]{hyperref}
\hypersetup{%
  pdfborder = {0 0 0}, % Comment this line if you want a RED box around links
  plainpages=false,%
  pdfauthor={Author(s)},%
  pdftitle={Title},%
  pdfsubject={Subject},%
  bookmarksnumbered=true,%
  colorlinks=false,%
  citecolor=black,%
  filecolor=black,%
  linkcolor=black,% This changes the color of the link - e.g. blue
  urlcolor=black,%
  pdfstartview=FitH%
}

%%%%%%%%%%%%%%%%%%%%%%%%%%%%%%%%%%%%%%%%%%%%%%%%
%%%               ! Packages !               %%%
% New packages should be placed in the file for
% packages. 
%%%%%%%%%%%%%%%%%%%%%%%%%%%%%%%%%%%%%%%%%%%%%%%%%
%                                              %
%                Extra packages                %
%                                              %
%%%%%%%%%%%%%%%%%%%%%%%%%%%%%%%%%%%%%%%%%%%%%%%%
% Insert your new packages here




%%%%%%%%%%%%%%%%%%%%%%%%%%%%%%%%%%%%%%%%%%%%%%%%
% Graphics and Tables
% http://en.wikibooks.org/wiki/LaTeX/Importing_Graphics
% http://en.wikibooks.org/wiki/LaTeX/Tables
% http://en.wikibooks.org/wiki/LaTeX/Colors
%%%%%%%%%%%%%%%%%%%%%%%%%%%%%%%%%%%%%%%%%%%%%%%%

% Load a colour package
\usepackage{xcolor}
\definecolor{aaublue}{RGB}{33,26,82} % Dark blue

% Gives the possibility to use colors by names.
% It can also change the color of text or text
% backgrounds.
\usepackage{color}

% Makes easier highlights/text background available
\usepackage{soul, soulutf8}

% The standard graphics inclusion package
\usepackage{graphicx}

% Set up how figure and table captions are displayed
\usepackage{caption}

% Makes it possible to have multiple figures next to each other
\usepackage{subcaption}
\captionsetup{%
  font=footnotesize,    % Set font size to footnotesize
  labelfont=bf          % Bold label (e.g., Figure 3.2) font
}

% Make the standard latex tables look so much better
\usepackage{array,booktabs}

% Enable the use of frames around, e.g., theorems
% The framed package is used in the example environment
\usepackage{framed}

% Adds support for full page background picture
\usepackage[contents={},color=gray]{background} % Legacy!!

% Maybe not necessary
\usepackage{wrapfig}

% Allows for cells spanning multiple rows in a table
\usepackage{multirow}

% Used for multiple columns
\usepackage{multicol}

% Advanced tables
\usepackage{tabularx}

% Width regulations
\usepackage{adjustbox}
\usepackage{array}
\usepackage{booktabs}


%%%%%%%%%%%%%%%%%%%%%%%%%%%%%%%%%%%%%%%%%%%%%%%%
% Mathematics
% http://en.wikibooks.org/wiki/LaTeX/Mathematics
%%%%%%%%%%%%%%%%%%%%%%%%%%%%%%%%%%%%%%%%%%%%%%%%

% Defines new environments such as equation,
% align and split 
\usepackage{amsmath}

% Adds new math symbols
\usepackage{amssymb}

% Use theorems in your document
% The ntheorem package is also used for the example environment
% When using thmmarks, amsmath must be an option as well. Otherwise \eqref doesn't work anymore.
\usepackage[framed,amsmath,thmmarks]{ntheorem}
\newenvironment{inlineCode}{\fontfamily{pcr}\selectfont}{\par}


%%%%%%%%%%%%%%%%%%%%%%%%%%%%%%%%%%%%%%%%%%%%%%%%
% Code listing
%%%%%%%%%%%%%%%%%%%%%%%%%%%%%%%%%%%%%%%%%%%%%%%%

% Can be used to insert code in the paper - 
% see the file '!CODE Template' for simple 
% instructions on how to use
\usepackage{listings}

% Using the package 'xcolor' to define colors
\definecolor{mGreen}{rgb}{0,0.6,0}
\definecolor{mGray}{rgb}{0.5,0.5,0.5}
\definecolor{mPurple}{rgb}{0.58,0,0.82}
\definecolor{backgroundColour}{rgb}{0.95,0.95,0.92}

\lstdefinestyle{JSStyle}{
    backgroundcolor=\color{backgroundColour},
    breakatwhitespace=false,
    breaklines=true,
    captionpos=b,
    keepspaces=true,
    numbers=left,
    numbersep=5pt,
    showspaces=false,
    showstringspaces=false,
    showtabs=false,
    tabsize=2,
    numberstyle=\tiny\color{mGray},
    commentstyle=\color{mGreen},
    keywordstyle=\color{magenta},
    stringstyle=\color{mPurple},
    basicstyle=\footnotesize \ttfamily,
    language=Java
}

\lstdefinestyle{CSharpStyle}{
    language=Java,
    basicstyle=\footnotesize \ttfamily,
    backgroundcolor=\color{backgroundColour},
    numbers=left,
    numberstyle=\tiny\color{mGray},
    numbersep=5pt,
    tabsize=2,
    breaklines=true,
    captionpos=b,
    breakatwhitespace=false,
    keepspaces=true,
    showspaces=false,
    showstringspaces=false,
    showtabs=false,
    commentstyle=\color{green},
    keywordstyle=\color{cyan},
    stringstyle=\color{blue},
    identifierstyle=\color{red},
    morekeywords={ abstract, event, struct,
    as, explicit, null, switch,
    base, extern, object, this,
    bool, false, operator, throw,
    break, finally, out, true,
    byte, fixed, override, try,
    case, float, params, typeof,
    catch, for, private, uint,
    char, foreach, protected, ulong,
    checked, goto, public, unchecked,
    class, if, readonly, unsafe,
    const, implicit, ref, ushort,
    continue, in, return, using,
    decimal, int, sbyte, virtual,
    default, interface, sealed, volatile,
    delegate, internal, short, void,
    do, is, sizeof, while,
    double, lock, stackalloc,
    else, long, static, enum, namespace, new, 
    string, await, Assert, var, async, Task}
}


%%%%%%%%%%%%%%%%%%%%%%%%%%%%%%%%%%%%%%%%%%%%%%%%
% Pseudocode
%%%%%%%%%%%%%%%%%%%%%%%%%%%%%%%%%%%%%%%%%%%%%%%%

% These packages allows the possibility to input
% pseudocode - see the file '!PSEDOCODE Help' 
% for simple instruction on how to use
\usepackage{algorithm}
\usepackage{algpseudocode}
\usepackage{program}

\algnewcommand{\algvar}{\texttt}
\algnewcommand{\assign}{\leftarrow}
\algnewcommand{\NIL}{\texttt{NIL}}
\algnewcommand{\NULL}{\texttt{NULL}}


%%%%%%%%%%%%%%%%%%%%%%%%%%%%%%%%%%%%%%%%%%%%%%%%
% TikZ
%%%%%%%%%%%%%%%%%%%%%%%%%%%%%%%%%%%%%%%%%%%%%%%%

% Tikz can be used to generate diagrams and figures
\usepackage{tikz}
\usetikzlibrary{fit, calc, positioning, shapes.geometric, arrows}

\tikzstyle{b} = [rectangle, draw, fill=white!20, node distance=8em, text width=6em, text centered, rounded corners, minimum height=4em, thick]
\tikzstyle{c} = [rectangle, draw, inner sep=0.5cm, dashed]
\tikzstyle{d} = [node distance=0, minimum height=1.5em]
\tikzstyle{g} = [circle, draw, fill=white!5, very thick, minimum size=3em]
\tikzstyle{l} = [draw, -latex',thick, line width=.2em]
\tikzstyle{ll} = [draw, latex'-latex',thick, line width=.2em]
\tikzstyle{p} = [draw,thick]
\tikzstyle{tab} = [midway, text width=10em, align=center]
\tikzstyle{ta} = [tab, above=1.5em]
\tikzstyle{tb} = [tab, below=1.5em]
%%%%%%%%%%%%%%%%%%%%%%%%%%%%%%%%%%%%%%%%%%%%%%%%     % Set up of the document - do not touch unless you know what you are doing!
\newacronym{ex}{expl}{example} % to use \gls{ex}     % Adding the acronyms.
%%%%%%%%%%%%%%%%%%%%%%%%%%%%%%%%%%%%%%%%%%%%%%%%
%                                              %
%                Extra packages                %
%                                              %
%%%%%%%%%%%%%%%%%%%%%%%%%%%%%%%%%%%%%%%%%%%%%%%%
% Insert your new packages here




%%%%%%%%%%%%%%%%%%%%%%%%%%%%%%%%%%%%%%%%%%%%%%%%
% Graphics and Tables
% http://en.wikibooks.org/wiki/LaTeX/Importing_Graphics
% http://en.wikibooks.org/wiki/LaTeX/Tables
% http://en.wikibooks.org/wiki/LaTeX/Colors
%%%%%%%%%%%%%%%%%%%%%%%%%%%%%%%%%%%%%%%%%%%%%%%%

% Load a colour package
\usepackage{xcolor}
\definecolor{aaublue}{RGB}{33,26,82} % Dark blue

% Gives the possibility to use colors by names.
% It can also change the color of text or text
% backgrounds.
\usepackage{color}

% Makes easier highlights/text background available
\usepackage{soul, soulutf8}

% The standard graphics inclusion package
\usepackage{graphicx}

% Set up how figure and table captions are displayed
\usepackage{caption}

% Makes it possible to have multiple figures next to each other
\usepackage{subcaption}
\captionsetup{%
  font=footnotesize,    % Set font size to footnotesize
  labelfont=bf          % Bold label (e.g., Figure 3.2) font
}

% Make the standard latex tables look so much better
\usepackage{array,booktabs}

% Enable the use of frames around, e.g., theorems
% The framed package is used in the example environment
\usepackage{framed}

% Adds support for full page background picture
\usepackage[contents={},color=gray]{background} % Legacy!!

% Maybe not necessary
\usepackage{wrapfig}

% Allows for cells spanning multiple rows in a table
\usepackage{multirow}

% Used for multiple columns
\usepackage{multicol}

% Advanced tables
\usepackage{tabularx}

% Width regulations
\usepackage{adjustbox}
\usepackage{array}
\usepackage{booktabs}


%%%%%%%%%%%%%%%%%%%%%%%%%%%%%%%%%%%%%%%%%%%%%%%%
% Mathematics
% http://en.wikibooks.org/wiki/LaTeX/Mathematics
%%%%%%%%%%%%%%%%%%%%%%%%%%%%%%%%%%%%%%%%%%%%%%%%

% Defines new environments such as equation,
% align and split 
\usepackage{amsmath}

% Adds new math symbols
\usepackage{amssymb}

% Use theorems in your document
% The ntheorem package is also used for the example environment
% When using thmmarks, amsmath must be an option as well. Otherwise \eqref doesn't work anymore.
\usepackage[framed,amsmath,thmmarks]{ntheorem}
\newenvironment{inlineCode}{\fontfamily{pcr}\selectfont}{\par}


%%%%%%%%%%%%%%%%%%%%%%%%%%%%%%%%%%%%%%%%%%%%%%%%
% Code listing
%%%%%%%%%%%%%%%%%%%%%%%%%%%%%%%%%%%%%%%%%%%%%%%%

% Can be used to insert code in the paper - 
% see the file '!CODE Template' for simple 
% instructions on how to use
\usepackage{listings}

% Using the package 'xcolor' to define colors
\definecolor{mGreen}{rgb}{0,0.6,0}
\definecolor{mGray}{rgb}{0.5,0.5,0.5}
\definecolor{mPurple}{rgb}{0.58,0,0.82}
\definecolor{backgroundColour}{rgb}{0.95,0.95,0.92}

\lstdefinestyle{JSStyle}{
    backgroundcolor=\color{backgroundColour},
    breakatwhitespace=false,
    breaklines=true,
    captionpos=b,
    keepspaces=true,
    numbers=left,
    numbersep=5pt,
    showspaces=false,
    showstringspaces=false,
    showtabs=false,
    tabsize=2,
    numberstyle=\tiny\color{mGray},
    commentstyle=\color{mGreen},
    keywordstyle=\color{magenta},
    stringstyle=\color{mPurple},
    basicstyle=\footnotesize \ttfamily,
    language=Java
}

\lstdefinestyle{CSharpStyle}{
    language=Java,
    basicstyle=\footnotesize \ttfamily,
    backgroundcolor=\color{backgroundColour},
    numbers=left,
    numberstyle=\tiny\color{mGray},
    numbersep=5pt,
    tabsize=2,
    breaklines=true,
    captionpos=b,
    breakatwhitespace=false,
    keepspaces=true,
    showspaces=false,
    showstringspaces=false,
    showtabs=false,
    commentstyle=\color{green},
    keywordstyle=\color{cyan},
    stringstyle=\color{blue},
    identifierstyle=\color{red},
    morekeywords={ abstract, event, struct,
    as, explicit, null, switch,
    base, extern, object, this,
    bool, false, operator, throw,
    break, finally, out, true,
    byte, fixed, override, try,
    case, float, params, typeof,
    catch, for, private, uint,
    char, foreach, protected, ulong,
    checked, goto, public, unchecked,
    class, if, readonly, unsafe,
    const, implicit, ref, ushort,
    continue, in, return, using,
    decimal, int, sbyte, virtual,
    default, interface, sealed, volatile,
    delegate, internal, short, void,
    do, is, sizeof, while,
    double, lock, stackalloc,
    else, long, static, enum, namespace, new, 
    string, await, Assert, var, async, Task}
}


%%%%%%%%%%%%%%%%%%%%%%%%%%%%%%%%%%%%%%%%%%%%%%%%
% Pseudocode
%%%%%%%%%%%%%%%%%%%%%%%%%%%%%%%%%%%%%%%%%%%%%%%%

% These packages allows the possibility to input
% pseudocode - see the file '!PSEDOCODE Help' 
% for simple instruction on how to use
\usepackage{algorithm}
\usepackage{algpseudocode}
\usepackage{program}

\algnewcommand{\algvar}{\texttt}
\algnewcommand{\assign}{\leftarrow}
\algnewcommand{\NIL}{\texttt{NIL}}
\algnewcommand{\NULL}{\texttt{NULL}}


%%%%%%%%%%%%%%%%%%%%%%%%%%%%%%%%%%%%%%%%%%%%%%%%
% TikZ
%%%%%%%%%%%%%%%%%%%%%%%%%%%%%%%%%%%%%%%%%%%%%%%%

% Tikz can be used to generate diagrams and figures
\usepackage{tikz}
\usetikzlibrary{fit, calc, positioning, shapes.geometric, arrows}

\tikzstyle{b} = [rectangle, draw, fill=white!20, node distance=8em, text width=6em, text centered, rounded corners, minimum height=4em, thick]
\tikzstyle{c} = [rectangle, draw, inner sep=0.5cm, dashed]
\tikzstyle{d} = [node distance=0, minimum height=1.5em]
\tikzstyle{g} = [circle, draw, fill=white!5, very thick, minimum size=3em]
\tikzstyle{l} = [draw, -latex',thick, line width=.2em]
\tikzstyle{ll} = [draw, latex'-latex',thick, line width=.2em]
\tikzstyle{p} = [draw,thick]
\tikzstyle{tab} = [midway, text width=10em, align=center]
\tikzstyle{ta} = [tab, above=1.5em]
\tikzstyle{tb} = [tab, below=1.5em]     % New packages should be included here
\input{setup/hyphenations.tex} % How to split words
\input{setup/macros.tex}       % Macros for making the frontpage
%%%%%%%%%%%%%%%%%%%%%%%%%%%%%%%%%%%%%%%%%%%%%%%%

% Highlight text and add a todo
% Use: "\hltodo{Here goes the text you want to make a comment to}{Here goes your notes to the text}"
\newcommand{\hltodo}[2]{\texthl{#1}\todo{#2}}

%%%%%%%%%%%%%%%%%%%%%%%%%%%%%%%%%%%%%%%%%%%%%%%%

% Make an indented quote
% NOTE: It is needed to insert a "\cite{}" as the second parameter
% Use: "\indentquote{Here goes the quote}{\cite{Here goes the cite}}"

\newcommand{\indentquote}[3]
{
\begin{center}
    \begin{minipage}{0.75\textwidth}
        \textit{``#1''}~#2
    \end{minipage}
\end{center}
}

% Rotate text and make bold
% Use: "\rbf{Here goes the rotation degree (use a number)}{Here goes the text}"

\newcommand{\rbf}[2]{
    \rotatebox{#1}{\textbf{#2}}
    }
     % Custom commands

%%%%%%%%%%%%%%%%%%%%%%%%%%%%%%%%%%%%%
% Group information
%%%%%%%%%%%%%%%%%%%%%%%%%%%%%%%%%%%%%
% This file is for making the variables
% which are used for inserting information
% in the frontpages, titlepage and preface.
\newtoks\projectTitle
\newtoks\projectSubtitle
\newtoks\groupNumber
\newtoks\supervisor
\newtoks\projectTheme
\newtoks\projectPeriod
\newtoks\semester
\newtoks\projectAbstract

\newtoks\groupMemberOne
\newtoks\groupMemberTwo
\newtoks\groupMemberThree
\newtoks\groupMemberFour
\newtoks\groupMemberFive
\newtoks\groupMemberSix
\newtoks\groupMemberSeven

% Insert your information in the following boxes.
% You may leave them empty.

% If not 7 members in group just leave the the
% groupmembers empty from the buttom up.
\groupMemberOne   = {Insert your name} % ie. John Doe
\groupMemberTwo   = {John Doe} % ie. John Doe
\groupMemberThree = {} % ie. John Doe
\groupMemberFour  = {} % ie. John Doe
\groupMemberFive  = {} % ie. John Doe
\groupMemberSix   = {} % ie. John Doe
\groupMemberSeven = {} % ie. John Doe

\projectTitle     = {Insert project title here}
\projectSubtitle  = {Insert project subtitle here or leave empty}
\groupNumber      = {Insert group number here} % ie. 3.1.24
\supervisor       = {Insert supervisor name here} % ie. John Doe
\projectTheme     = {Insert project theme here} % ie. How to optimise LaTeX
\projectPeriod    = {Insert project period here} % ie. Spring 2023
\semester         = {Insert your semestser} % ie. 2. semester
\projectAbstract  = {
    Insert your abstract here.
    It will apear in the abstract section on the titlepage.
} % You can make linebreaks within the {}


%%%%%%%%%%%%%%%%%%
%%% Change here for one- 
%%% or two-page setup
\setlength{\marginparwidth}{2cm} % !!If using two pages-setup (book-style) mark this line as comment
%%%%%%%%%%%%%%%%%%

% Begins the document for text
\begin{document}


%%%%%%%%%%%%%%%%%%%%%%%%%%%%%%%%%
%%%%%%%%%% Frontmatter %%%%%%%%%%
% Setup for pages before content of the paper

\pagestyle{empty}     % Disable headers and footers
\pagenumbering{roman} % Use roman page numbering in the frontmatter


%%%%%%%%%%%%%% Front pages %%%%%%%%%%%%%
% Here you can choose which of the frontpages
% you would like to use for your project paper.
% You can delete or comment the frontpages you
% are not going to use.

\pdfbookmark[0]{Front page}{label:frontpage-simple}%

\begin{titlepage}
  \addtolength{\hoffset}{0.5\evensidemargin-0.5\oddsidemargin} %set equal margins on the frontpage - remove this line if you want default margins
  \noindent%
  \begin{tabular}{@{}p{\textwidth}@{}}
    \toprule[2pt]
    \midrule
    \vspace{0.2cm}
    \begin{center}
    \Huge{\textbf{
        % Insert your title here
        \the\projectTitle
    }}
    \end{center}
    \begin{center}
      \Large{
        % Insert your subtitle here
        \the\projectSubtitle
      }
    \end{center}
    \vspace{0.2cm}\\
    \midrule
    \toprule[2pt]
  \end{tabular}
  \vspace{4 cm}
  \begin{center}
    {\large
        %Insert document type (e.g., Project Report)
        Project Paper
    }\\
    \vspace{0.2cm}
    {\Large
        %Insert your group name or real names here
        \the\groupNumber
    }
  \end{center}
  \vfill
  \begin{center}
  Aalborg University\\
  Computer Science
  \end{center}
\end{titlepage}
\clearpage
\cleardoublepage
\pdfbookmark[0]{Front page}{label:frontpage-aau}%

\begin{titlepage}
\vspace*{\fill}
    \backgroundsetup{ % legacy!!
    scale=1.1,
    angle=0,
    opacity=1.0,  %% Adjust
    contents={\includegraphics[width=\paperwidth,height=\paperheight]{AAUgraphics/aau_waves}}
    }
  \addtolength{\hoffset}{0.5\evensidemargin-0.5\oddsidemargin} % Set equal margins on the frontpage - comment this line if you want default margins
  \noindent%
  {\color{white}\fboxsep0pt\colorbox{aaublue}{\begin{tabular}{@{}p{\textwidth}@{}}
    \begin{center}
    \Huge{\textbf{
        % Your title here
        \the\projectTitle
    }}
    \end{center}
    \begin{center}
      \Large{
        % Your subtitle here
        \the\projectSubtitle
      }
    \end{center}
    \vspace{0.2cm}
   \begin{center}
    {\Large
        % Names separated by comma
        \if \the\groupMemberOne ""
        \else
        \the\groupMemberOne
        \fi
        \if \the\groupMemberTwo ""
        \else
        , \the\groupMemberTwo
        \fi
        \if \the\groupMemberThree ""
        \else
        , \the\groupMemberThree
        \fi
        \if \the\groupMemberFour ""
        \else
        , \the\groupMemberFour
        \fi
        \if \the\groupMemberFive ""
        \else
        , \the\groupMemberFive
        \fi
        \if \the\groupMemberSix ""
        \else
        , \the\groupMemberSix
        \fi
        \if \the\groupMemberSeven ""
        \else
        , \the\groupMemberSeven
        \fi
    }\\
    \vspace{0.2cm}
    {\large
      Department of Computer Science, DAT-\{\the\semester\}, \the\year-\the\month-\the\day % Insert semester
    }
   \end{center}
   %\vspace{0.2cm}
%% Comment this section in if you are doing Bachelor or Master Project   
   %\begin{center}
    %{\Large
      % Master's Project
      % Bachelor Project
    %}
   %\end{center}
  \end{tabular}}}
  \vfill
  \begin{center}
    %\includegraphics[width=0.2\paperwidth]{AAUgraphics/aau_logo_circle_en} % Comment this line in for English version
    \includegraphics[width=0.2\paperwidth]{AAUgraphics/aau_logo_circle_da}  % Comment this line in for Danish version
  \end{center}
\end{titlepage}
\clearpage

\cleardoublepage
\input{sections/0frontpages/image-frontpage.tex}
\cleardoublepage
% ENGLISH %

\pdfbookmark[0]{English title page}{label:titlepage_en}
\aautitlepage{%
    \englishprojectinfo{
        % Title
        \the\projectTitle
    }{%
        % Theme
        \the\projectTheme
    }{%
        % Project period
        \the\projectPeriod
    }{%
        % Project group
        \the\groupNumber
    }{%
        % List of group members
        \if \the\groupMemberOne ""
        \else
        \the\groupMemberOne
        \fi
        \if \the\groupMemberTwo ""
        \else
        \\\the\groupMemberTwo
        \fi
        \if \the\groupMemberThree ""
        \else
        \\\the\groupMemberThree
        \fi
        \if \the\groupMemberFour ""
        \else
        \\\the\groupMemberFour
        \fi
        \if \the\groupMemberFive ""
        \else
        \\\the\groupMemberFive
        \fi
        \if \the\groupMemberSix ""
        \else
        \\\the\groupMemberSix
        \fi
        \if \the\groupMemberSeven ""
        \else
        \\\the\groupMemberSeven
        \fi
    }{%
        % List of supervisors
        \the\supervisor
    }{%
        1 % Number of printed copies
    }{%
        \today % Date of completion
    }%
}{% Department and address
 \textbf{Department of Computer Science}\\
 Aalborg University\\
 \href{http://www.aau.dk}{http://www.aau.dk}
}{% The abstract
\the\projectAbstract
}

% DANSK %

%\cleardoublepage
% {\selectlanguage{danish}
% \pdfbookmark[0]{Danish title page}{label:titlepage_da}
% \aautitlepage{%
%   \danishprojectinfo{
%     % Title
%     Indsæt titel
%   }{%
%     % Theme
%     Indsæt tema
%   }{%
%     % Project period 
%     Indsæt periode
%   }{%
%     % Project group
%     Indsæt gruppenummer/-navn
%   }{%
%     % List of group members
%     Indsæt gruppemedlemmers navne
%   }{%
%     % List of supervisors
%     Indsæt vejleder navne
%   }{%
%     1 % Number of printed copies
%   }{%
%     \today % Date of completion
%   }%
% }{% Department and address
%   \textbf{IT og Design}\\
%   Aalborg Universitet\\
%   \href{http://www.aau.dk}{http://www.aau.dk}
% }{% The abstract
% Indsæt abstract  
% }}

\cleardoublepage

% Preface with signatures
% It is recommended to remove this from
% the paper since it has no real use.
% NOTE: You have to manually change the
% name and email in the file.
\chapter*{Preface\markboth{Preface}{Preface}}
\label{ch:preface}
%\addcontentsline{toc}{chapter}{Preface}

\vspace{3\baselineskip}
\begin{minipage}[b]{0.45\textwidth}
 \centering
 \rule{\textwidth}{0.5pt}\\
  Navn \#1\\
 {\footnotesize Navn \#1@student.aau.dk}
\end{minipage}
\hfill
\begin{minipage}[b]{0.45\textwidth}
 \centering
 \rule{\textwidth}{0.5pt}\\
 Navn \#2\\
 {\footnotesize Navn \#2@student.aau.dk}
\end{minipage}

\vspace{3\baselineskip}
\begin{minipage}[b]{0.45\textwidth}
 \centering
 \rule{\textwidth}{0.5pt}\\
 Navn \#3\\
 {\footnotesize Navn \#3@student.aau.dk}
\end{minipage}
\hfill
\begin{minipage}[b]{0.45\textwidth}
 \centering
 \rule{\textwidth}{0.5pt}\\
 Navn \#4\\
 {\footnotesize Navn \#4@student.aau.dk}
\end{minipage}

\vspace{3\baselineskip}
\begin{minipage}[b]{0.45\textwidth}
 \centering
 \rule{\textwidth}{0.5pt}\\
 Navn \#5\\
 {\footnotesize Navn \#5@student.aau.dk}
\end{minipage}
\hfill
\begin{minipage}[b]{0.45\textwidth}
 \centering
 \rule{\textwidth}{0.5pt}\\
 Navn \#6\\
 {\footnotesize Navn \#6@student.aau.dk}
\end{minipage}

\vspace{3\baselineskip}

\begin{center}
\begin{minipage}[b]{0.45\textwidth}
 \centering
 \rule{\textwidth}{0.5pt}
 Navn \#7\\
 {\footnotesize Navn \#7@student.aau.dk}
\end{minipage}
\end{center} % Comment this line to remove Preface with signatures
\cleardoublepage

\pdfbookmark[0]{Contents}{label:contents}
\pagestyle{fancy} % Enable headers and footers again

% Insert table of contents
\tableofcontents

% Inserts a list of ToDos
\listoftodos % Comment this to remove ToDos from paper


%%%%%%%%%%%%%%%%%%%%%%%%%%%%%%%%
%%%%%%%%%% Mainmatter %%%%%%%%%%
% Setup for the main content of the paper

% If you do not write the report in English,
% translating the bibliography title to, e.g., Danish,
% should not be done here. Instead, you should change the
% main language in the preamble (look for the line \usepackage[english]{babel}
% and change the 'english' option to danish). See more in the babel package documentation.
\pagenumbering{arabic} % Use arabic page numbering in the mainmatter


%%%%%%%%%%%%%%% Chapters %%%%%%%%%%%%%%%
% Use this file to input all chapters
% Chapter 1
\chapter{Introduction}
    \input{sections/1introduction/introduction}

% Chapter 2
\input{sections/2problemAnalysis/!Header_problemAnalysis}

% Chapter 3
\input{sections/3problemStatement/!Header_problemStatement}

% Chapter 4
\input{sections/4design/!Header_design}

% Chapter 5
\input{sections/5product/!Header_product}

% Chapter 6
\input{sections/6discussion/!Header_discussion}

% Chapter 7
\chapter{Conclusion}
    \input{sections/7conclusion/Conclusion}

%%%%%%%%%%%%%%% Acronyms %%%%%%%%%%%%%%%
% This prints the acronyms used in the paper.
\printglossary[type=\acronymtype]


%%%%%%%%%%%%% Bibliography %%%%%%%%%%%%%
% If you have multiple .bib files, remember to
% include them in the ./setup/preamble.tex file
\printbibliography[heading=bibintoc, title=Bibliography]
\label{bib:mybiblio}


%%%%%%%%%%%%%%% Appendix %%%%%%%%%%%%%%%
\appendix
% Use this file to input the material for the appendix
\chapter{Work contract}\label{ch:appAlabel}


% End of document
\end{document}
