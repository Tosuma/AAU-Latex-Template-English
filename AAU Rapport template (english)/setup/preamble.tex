%%%%%%%%%%%%%%%%%%%%%%%%%%%%%%%%%%%%%%%%%%%%%%%%
%      DO NOT TOUCH ANYTHING IN THIS FILE      %
%      UNLESS YOU KNOW WHAT YOU ARE DOING      %
%                                              %
%        New packages should be included       %
%             in the file 'packages'           %
%%%%%%%%%%%%%%%%%%%%%%%%%%%%%%%%%%%%%%%%%%%%%%%%


\documentclass[11pt,oneside,a4paper,openright]{report}
%%%%%%%%%%%%%%%%%%%%%%%%%%%%%%%%%%%%%%%%%%%%%%%%
% Language, Encoding and Fonts
% http://en.wikibooks.org/wiki/LaTeX/Internationalization
%%%%%%%%%%%%%%%%%%%%%%%%%%%%%%%%%%%%%%%%%%%%%%%%
% Select encoding of your inputs. Depends on
% your operating system and its default input
% encoding. Typically, you should use
%   Linux  : utf8 (most modern Linux distributions)
%            latin1 
%   Windows: ansinew
%            latin1 (works in most cases)
%   Mac    : applemac
% Notice that you can manually change the input
% encoding of your files by selecting "save as"
% an select the desired input encoding.
\usepackage[utf8]{inputenc}

% Make latex understand and use the typographic
% rules of the language used in the document.
%%% Change this to danish for danish typesettings %%%
%%% (automatically generated text)                %%%
\usepackage[english]{babel}

% Use the palatino font
\usepackage[sc]{mathpazo}
\linespread{1.05} % Palatino needs more leading (space between lines)

% Choose the font encoding
\usepackage[T1]{fontenc}


%%%%%%%%%%%%%%%%%%%%%%%%%%%%%%%%%%%%%%%%%%%%%%%%
% Page Layout
% http://en.wikibooks.org/wiki/LaTeX/Page_Layout
%%%%%%%%%%%%%%%%%%%%%%%%%%%%%%%%%%%%%%%%%%%%%%%%
% Change margins, papersize, etc of the document
\usepackage[
  inner=28mm, % Left margin on an odd page
  outer=41mm, % Right margin on an odd page
]{geometry}

% Modify how \chapter, \section, etc. look
% The titlesec package is very configureable
\usepackage{titlesec}
\titleformat{\chapter}[display]{\normalfont\huge\bfseries}{\chaptertitlename\ \thechapter}{20pt}{\Huge}
\titleformat*{\section}{\normalfont\Large\bfseries}
\titleformat*{\subsection}{\normalfont\large\bfseries}
\titleformat*{\subsubsection}{\normalfont\normalsize\bfseries}
%\titleformat*{\paragraph}{\normalfont\normalsize\bfseries}
%\titleformat*{\subparagraph}{\normalfont\normalsize\bfseries}

% Setting level of numbering (default for "report" is 3)
% With '-1' you have non number also for chapters
\setcounter{secnumdepth}{3}

% Clear empty pages between chapters
\let\origdoublepage\cleardoublepage
\newcommand{\clearemptydoublepage}{%
  \clearpage
  {\pagestyle{empty}\origdoublepage}%
}
\let\cleardoublepage\clearemptydoublepage

% Change the headers and footers
\usepackage{fancyhdr}
\pagestyle{fancy}
\fancyhf{} % Delete everything
\renewcommand{\headrulewidth}{0pt} % Remove the horizontal line in the header

%%%%%%%%%%%%%%%%%%
%%% Onepage setup - with fixed position for page number and text

\fancyhead[L]{\small\nouppercase\leftmark}
\fancyhead[R]{\thepage}

%%% Twopage setup (book-style) with alternating position
%%% for page number and text 
%%% !!REMEMBER!! To change margin settings in "master.tex"

% \fancyhead[RE]{\small\nouppercase\leftmark} %even page - chapter title
% \fancyhead[LO]{\small\nouppercase\rightmark} %uneven page - section title
% \fancyhead[LE,RO]{\thepage} %page number on all pages
%%%%%%%%%%%%%%%%%%

% Do not stretch the content of a page. Instead,
% insert white space at the bottom of the page
\raggedbottom

% Enable arithmetics with length. Useful when
% typesetting the layout.
\usepackage{calc}


%%%%%%%%%%%%%%%%%%%%%%%%%%%%%%%%%%%%%%%%%%%%%%%%
% Bibliography
% http://en.wikibooks.org/wiki/LaTeX/Bibliography_Management
%%%%%%%%%%%%%%%%%%%%%%%%%%%%%%%%%%%%%%%%%%%%%%%%

% Adds the bibliography used for citing
\usepackage[backend=biber,
  bibencoding=utf8,
  style=numeric-comp
]{biblatex}
\addbibresource{bib/mybib}


%%%%%%%%%%%%%%%%%%%%%%%%%%%%%%%%%%%%%%%%%%%%%%%%
% Misc
%%%%%%%%%%%%%%%%%%%%%%%%%%%%%%%%%%%%%%%%%%%%%%%%

% Add bibliography and index to the table of
% contents
\usepackage[nottoc]{tocbibind}

% Add the command \pageref{LastPage} which refers to the
% page number of the last page
\usepackage{lastpage}

% Add todo notes in the margin of the document
\usepackage[
  %  disable, %turn off todonotes
  colorinlistoftodos, %enable a coloured square in the list of todos
  textwidth=\marginparwidth, %set the width of the todonotes
  textsize=scriptsize, %size of the text in the todonotes
]{todonotes}

% Add quotes
\usepackage{csquotes}


%%%%%%%%%%%%%%%%%%%%%%%%%%%%%%%%%%%%%%%%%%%%%%%%
% Hyperlinks
% http://en.wikibooks.org/wiki/LaTeX/Hyperlinks
%%%%%%%%%%%%%%%%%%%%%%%%%%%%%%%%%%%%%%%%%%%%%%%%

% Enable hyperlinks and insert info into the pdf
% file. Hypperref should be loaded as one of the 
% last packages
\usepackage[pdfpagelabels]{hyperref}
\hypersetup{%
  pdfborder = {0 0 0}, % Comment this line if you want a RED box around links
  plainpages=false,%
  pdfauthor={Author(s)},%
  pdftitle={Title},%
  pdfsubject={Subject},%
  bookmarksnumbered=true,%
  colorlinks=false,%
  citecolor=black,%
  filecolor=black,%
  linkcolor=black,% This changes the color of the link - e.g. blue
  urlcolor=black,%
  pdfstartview=FitH%
}

%%%%%%%%%%%%%%%%%%%%%%%%%%%%%%%%%%%%%%%%%%%%%%%%
%%%               ! Packages !               %%%
% New packages should be placed in the file for
% packages. 
%%%%%%%%%%%%%%%%%%%%%%%%%%%%%%%%%%%%%%%%%%%%%%%%%
%                                              %
%                Extra packages                %
%                                              %
%%%%%%%%%%%%%%%%%%%%%%%%%%%%%%%%%%%%%%%%%%%%%%%%
% Insert your new packages here




%%%%%%%%%%%%%%%%%%%%%%%%%%%%%%%%%%%%%%%%%%%%%%%%
% Graphics and Tables
% http://en.wikibooks.org/wiki/LaTeX/Importing_Graphics
% http://en.wikibooks.org/wiki/LaTeX/Tables
% http://en.wikibooks.org/wiki/LaTeX/Colors
%%%%%%%%%%%%%%%%%%%%%%%%%%%%%%%%%%%%%%%%%%%%%%%%

% Load a colour package
\usepackage{xcolor}
\definecolor{aaublue}{RGB}{33,26,82} % Dark blue

% Gives the possibility to use colors by names.
% It can also change the color of text or text
% backgrounds.
\usepackage{color}

% Makes easier highlights/text background available
\usepackage{soul, soulutf8}

% The standard graphics inclusion package
\usepackage{graphicx}

% Set up how figure and table captions are displayed
\usepackage{caption}

% Makes it possible to have multiple figures next to each other
\usepackage{subcaption}
\captionsetup{%
  font=footnotesize,    % Set font size to footnotesize
  labelfont=bf          % Bold label (e.g., Figure 3.2) font
}

% Make the standard latex tables look so much better
\usepackage{array,booktabs}

% Enable the use of frames around, e.g., theorems
% The framed package is used in the example environment
\usepackage{framed}

% Adds support for full page background picture
\usepackage[contents={},color=gray]{background} % Legacy!!

% Maybe not necessary
\usepackage{wrapfig}

% Allows for cells spanning multiple rows in a table
\usepackage{multirow}

% Used for multiple columns
\usepackage{multicol}

% Advanced tables
\usepackage{tabularx}

% Width regulations
\usepackage{adjustbox}
\usepackage{array}
\usepackage{booktabs}

% Advanced float placement
\usepackage{placeins}

%%%%%%%%%%%%%%%%%%%%%%%%%%%%%%%%%%%%%%%%%%%%%%%%
% Mathematics
% http://en.wikibooks.org/wiki/LaTeX/Mathematics
%%%%%%%%%%%%%%%%%%%%%%%%%%%%%%%%%%%%%%%%%%%%%%%%

% Defines new environments such as equation,
% align and split 
\usepackage{amsmath}

% Adds new math symbols
\usepackage{amssymb}

% Use theorems in your document
% The ntheorem package is also used for the example environment
% When using thmmarks, amsmath must be an option as well. Otherwise \eqref doesn't work anymore.
\usepackage[framed,amsmath,thmmarks]{ntheorem}
\newenvironment{inlineCode}{\fontfamily{pcr}\selectfont}{\par}


%%%%%%%%%%%%%%%%%%%%%%%%%%%%%%%%%%%%%%%%%%%%%%%%
% Code listing
%%%%%%%%%%%%%%%%%%%%%%%%%%%%%%%%%%%%%%%%%%%%%%%%

% Can be used to insert code in the paper - 
% see the file '!CODE Template' for simple 
% instructions on how to use
\usepackage{listings}

% Using the package 'xcolor' to define colors
\definecolor{mGreen}{rgb}{0,0.6,0}
\definecolor{mGray}{rgb}{0.5,0.5,0.5}
\definecolor{mPurple}{rgb}{0.58,0,0.82}
\definecolor{backgroundColour}{rgb}{0.95,0.95,0.92}

\lstdefinestyle{JSStyle}{
    backgroundcolor=\color{backgroundColour},
    breakatwhitespace=false,
    breaklines=true,
    captionpos=b,
    keepspaces=true,
    numbers=left,
    numbersep=5pt,
    showspaces=false,
    showstringspaces=false,
    showtabs=false,
    tabsize=2,
    numberstyle=\tiny\color{mGray},
    commentstyle=\color{mGreen},
    keywordstyle=\color{magenta},
    stringstyle=\color{mPurple},
    basicstyle=\footnotesize \ttfamily,
    language=Java
}

\lstdefinestyle{CSharpStyle}{
    language=Java,
    basicstyle=\footnotesize \ttfamily,
    backgroundcolor=\color{backgroundColour},
    numbers=left,
    numberstyle=\tiny\color{mGray},
    numbersep=5pt,
    tabsize=2,
    breaklines=true,
    captionpos=b,
    breakatwhitespace=false,
    keepspaces=true,
    showspaces=false,
    showstringspaces=false,
    showtabs=false,
    commentstyle=\color{green},
    keywordstyle=\color{cyan},
    stringstyle=\color{blue},
    identifierstyle=\color{red},
    morekeywords={ abstract, event, struct,
    as, explicit, null, switch,
    base, extern, object, this,
    bool, false, operator, throw,
    break, finally, out, true,
    byte, fixed, override, try,
    case, float, params, typeof,
    catch, for, private, uint,
    char, foreach, protected, ulong,
    checked, goto, public, unchecked,
    class, if, readonly, unsafe,
    const, implicit, ref, ushort,
    continue, in, return, using,
    decimal, int, sbyte, virtual,
    default, interface, sealed, volatile,
    delegate, internal, short, void,
    do, is, sizeof, while,
    double, lock, stackalloc,
    else, long, static, enum, namespace, new, 
    string, await, Assert, var, async, Task}
}


%%%%%%%%%%%%%%%%%%%%%%%%%%%%%%%%%%%%%%%%%%%%%%%%
% Code Blocks using Minted
%%%%%%%%%%%%%%%%%%%%%%%%%%%%%%%%%%%%%%%%%%%%%%%%

% This package gives the possibility to make
% code blocks in your text. % See the file
% '!CODE Template.tex' for instructions on how
% to use.
\usepackage[cache=false]{minted}


%%%%%%%%%%%%%%%%%%%%%%%%%%%%%%%%%%%%%%%%%%%%%%%%
% Pseudocode
%%%%%%%%%%%%%%%%%%%%%%%%%%%%%%%%%%%%%%%%%%%%%%%%

% These packages allows the possibility to input
% pseudocode - see the file '!PSEDOCODE Help' 
% for simple instruction on how to use
\usepackage{algorithm}
\usepackage{algpseudocode}
\usepackage{program}

\algnewcommand{\algvar}{\texttt}
\algnewcommand{\assign}{\leftarrow}
\algnewcommand{\NIL}{\texttt{NIL}}
\algnewcommand{\NULL}{\texttt{NULL}}


%%%%%%%%%%%%%%%%%%%%%%%%%%%%%%%%%%%%%%%%%%%%%%%%
% TikZ
%%%%%%%%%%%%%%%%%%%%%%%%%%%%%%%%%%%%%%%%%%%%%%%%

% Tikz can be used to generate diagrams and figures
\usepackage{tikz}
\usetikzlibrary{fit, calc, positioning, shapes.geometric, arrows}

\tikzstyle{b} = [rectangle, draw, fill=white!20, node distance=8em, text width=6em, text centered, rounded corners, minimum height=4em, thick]
\tikzstyle{c} = [rectangle, draw, inner sep=0.5cm, dashed]
\tikzstyle{d} = [node distance=0, minimum height=1.5em]
\tikzstyle{g} = [circle, draw, fill=white!5, very thick, minimum size=3em]
\tikzstyle{l} = [draw, -latex',thick, line width=.2em]
\tikzstyle{ll} = [draw, latex'-latex',thick, line width=.2em]
\tikzstyle{p} = [draw,thick]
\tikzstyle{tab} = [midway, text width=10em, align=center]
\tikzstyle{ta} = [tab, above=1.5em]
\tikzstyle{tb} = [tab, below=1.5em]
%%%%%%%%%%%%%%%%%%%%%%%%%%%%%%%%%%%%%%%%%%%%%%%%