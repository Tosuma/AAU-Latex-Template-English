\section{This is a section}
This is the text of the section.

\subsection{This is a sub section}
This is the text of the sub section.


\subsubsection{This is a sub sub section}
This is the text of the sub sub section.

\paragraph{This is a paragraph}
This is the text of the paragraph.

\section*{This is a different type of section}
If you add a "\texttt{*}" to the section, subsection or subsubsection command
it will not be shown in the table of content and it will not have any numbering.

\section{Another section}
As it can be seen the previous non-numbered section does not increment the
section numbering.


\section{How to do code examples}

Go to the file '\texttt{./!CODE Template.tex}' to get templates for coding blocks.

This is how to make code blocks with the \texttt{lstlisting} package:
\begin{figure}
\begin{lstlisting}[label={lst:code-block},caption={This is the caption of the
    code example},style=CSharpStyle]
#include <stdio.h>
int main() {
    // printf() displays the string inside quotation
    printf("Hello, World!");
    return 0;
}
\end{lstlisting}
\end{figure}

% NOTE: The Minted coding block does not currently work
% This is how to make code blocks with the \texttt{minted} package:

\begin{minted}{c}
#include <stdio.h>
int main() {
    // printf() displays the string inside quotation
    printf("Hello, World!");
    return 0;
}
\end{minted}

\subsection{How to make pseudocode blocks}

Go to the file '\texttt{./!PSEUDOCODE Help.tex}' to get templates for pseudocode blocks.

This is how to make pseudocode bloacks with the \texttt{algorithm} and
\texttt{algorithmic} packages. An overview of the commands can be found ind
\autoref{table:pseudocode-commands}.

\begin{algorithm}
    \caption{This is the caption/name of the pseudocode example}
    \label{alg:pseudocode}
    \begin{algorithmic}[1]
        \State{When event fires}
        \For{each eventNode.children}
            \State{\Call{doActionRecursive}{node}}
        \EndFor
        \Function{Async doActionRecursive}{node}
            \State{node.doAction}
            \For{each node.children}
                \State{\Call{doActionRecursive}{child}}
            \EndFor
        \EndFunction
    \end{algorithmic}
\end{algorithm}

\begin{table}[h]
    \begin{tabular}{r l}
        \textbackslash State\{\}              & Begin a statement or sentence in the code\\
        \hline

        \textbackslash While\{\}              & Begin while loop\\
        \textbackslash EndWhile               & End while loop\\
        \hline

        \textbackslash For\{\}                & Begin for loop\\
        \textbackslash To\{\}                 & From something to this\\
        \textbackslash EndFor                 & End for loop\\
        \hline

        \textbackslash If\{\}                 & Begin If statement\\
        \textbackslash ElsIf\{\}              & Begin an Else If statement\\
        \textbackslash Else\{\}               & Begin Else\\
        \textbackslash EndIf                  & End If statement\\
        \hline

        \textbackslash Then\{\}               & Then something\\
        \hline

        \textbackslash Function{name}{param}  & Make a function{name}{parameters}\\
        \textbackslash EndFunction            & End a function\\
        \textbackslash return\{\}             & Return something\\
        \textbackslash Call{name}{parameters} & Call a function\\

        \hline
        \textbackslash NIL                    & NIL\\
        \textbackslash Comment\{\}            & Add comment
    \end{tabular}
    \caption{Overview of the different pseudocode commands.}
    \label{table:pseudocode-commands}
\end{table}

\section{Acronyms}
This is an \gls{ex} of how use glossaries in your latex files. When using the
\texttt{\textbackslash gls\{\}} command you can insert abreviations and during compilation
the first abreviation will be shown in full while the rest will be in the
abreviated form. E.g., this is another \gls{ex} of the glossaries but the second
time.

NOTE: That this will likely only work if you compile locally.


\section{Floating elements}

In \LaTeX you can add floating elements, such as images, and \LaTeX will place
the picture automatically, however, it is possible to move the picture if it is
placed inside a \texttt{figure} environment
(\texttt{\textbackslash begin\{\}\dots\textbackslash end\{\}}).
A \texttt{figure} environment is a \texttt{float}, not variable type but a
floating element. This floating element can be placed by using square brackets,
e.g. \texttt{\textbackslash begin\{figure\}[h]}. \autoref{table:float-placement}
shows the different tags which can be used.

\begin{table}[h]
    \begin{tabular}{p{0.01\textwidth} p{0.9\textwidth}}
        h & Place the float here, i.e., approximately at the same point it occurs in the source text (however, not exactly at the spot)\\
        t & Position at the top of the page.\\
        b & Position at the bottom of the page.\\
        p & Put on a special page for floats only.\\
        ! & Override internal parameters LaTeX uses for determining "good" float positions.\\
        H & Places the float at precisely the location in the LaTeX code. Requires the float package, though may cause problems occasionally. This is somewhat equivalent to h!. 
    \end{tabular}
    \caption{Overview of the different tags to place floats.}
    \label{table:float-placement}
\end{table}

\subsection{Labels}
\subsubsection*{How to make a label}

When making figures, or other types of floats, it is possible to label them
which can be used to make references to the figure. This is done with the
\texttt{\textbackslash label\{\}} command inside the environment.
It is a good practise to begin the name of the label with the type of the
element which are labeled, e.g. \texttt{\textbackslash label\{eq:math-stuff\}}
for an equation, \texttt{\textbackslash label\{sec:interesting-section\}} for a
section, \texttt{\textbackslash label\{ch:design\}} for a chapter.
Once you have made a label it is wise to not alter it again, since you may miss
a place of referecence

\subsubsection*{How to refer to a label}

When you want to refer to a label use either the \texttt{\textbackslash ref\{\}}
or \texttt{\textbackslash autoref\{\}} command.
The \texttt{autoref} is recommended to use, since it also prints the type of the
label, e.g. here is a \texttt{\textbackslash ref\{\}}: \ref{table:float-placement} and
for the same label this is the \texttt{\textbackslash autoref\{\}}:
\autoref{table:float-placement}.



\subsection{Inserting Pictures}

To insert an image use the command \texttt{\textbackslash includegraphics\{\}}
and specify the location of the picture.

\begin{figure}[h]
    \centering
    \includegraphics{example-image-a}
\end{figure}

