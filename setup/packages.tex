%%%%%%%%%%%%%%%%%%%%%%%%%%%%%%%%%%%%%%%%%%%%%%%%
%                                              %
%                Extra packages                %
%                                              %
%%%%%%%%%%%%%%%%%%%%%%%%%%%%%%%%%%%%%%%%%%%%%%%%
% Insert your new packages here




%%%%%%%%%%%%%%%%%%%%%%%%%%%%%%%%%%%%%%%%%%%%%%%%
% Graphics and Tables
% http://en.wikibooks.org/wiki/LaTeX/Importing_Graphics
% http://en.wikibooks.org/wiki/LaTeX/Tables
% http://en.wikibooks.org/wiki/LaTeX/Colors
%%%%%%%%%%%%%%%%%%%%%%%%%%%%%%%%%%%%%%%%%%%%%%%%

% Load a colour package
\usepackage{xcolor}
\definecolor{aaublue}{RGB}{33,26,82} % Dark blue

% Gives the possibility to use colors by names.
% It can also change the color of text or text
% backgrounds.
\usepackage{color}

% Makes easier highlights/text background available
\usepackage{soul, soulutf8}

% The standard graphics inclusion package
\usepackage{graphicx}

% Set up how figure and table captions are displayed
\usepackage{caption}

% Makes it possible to have multiple figures next to each other
\usepackage{subcaption}
\captionsetup{%
  font=footnotesize,    % Set font size to footnotesize
  labelfont=bf          % Bold label (e.g., Figure 3.2) font
}

% Make the standard latex tables look so much better
\usepackage{array,booktabs}

% Enable the use of frames around, e.g., theorems
% The framed package is used in the example environment
\usepackage{framed}

% Adds support for full page background picture
\usepackage[contents={},color=gray]{background} % Legacy!!

% Maybe not necessary
\usepackage{wrapfig}

% Allows for cells spanning multiple rows in a table
\usepackage{multirow}

% Used for multiple columns
\usepackage{multicol}

% Advanced tables
\usepackage{tabularx}

% Width regulations
\usepackage{adjustbox}
\usepackage{array}
\usepackage{booktabs}


%%%%%%%%%%%%%%%%%%%%%%%%%%%%%%%%%%%%%%%%%%%%%%%%
% Mathematics
% http://en.wikibooks.org/wiki/LaTeX/Mathematics
%%%%%%%%%%%%%%%%%%%%%%%%%%%%%%%%%%%%%%%%%%%%%%%%

% Defines new environments such as equation,
% align and split 
\usepackage{amsmath}

% Adds new math symbols
\usepackage{amssymb}

% Use theorems in your document
% The ntheorem package is also used for the example environment
% When using thmmarks, amsmath must be an option as well. Otherwise \eqref doesn't work anymore.
\usepackage[framed,amsmath,thmmarks]{ntheorem}
\newenvironment{inlineCode}{\fontfamily{pcr}\selectfont}{\par}


%%%%%%%%%%%%%%%%%%%%%%%%%%%%%%%%%%%%%%%%%%%%%%%%
% Code listing
%%%%%%%%%%%%%%%%%%%%%%%%%%%%%%%%%%%%%%%%%%%%%%%%

% Can be used to insert code in the paper - 
% see the file '!CODE Template' for simple 
% instructions on how to use
\usepackage{listings}

% Using the package 'xcolor' to define colors
\definecolor{mGreen}{rgb}{0,0.6,0}
\definecolor{mGray}{rgb}{0.5,0.5,0.5}
\definecolor{mPurple}{rgb}{0.58,0,0.82}
\definecolor{backgroundColour}{rgb}{0.95,0.95,0.92}

\lstdefinestyle{JSStyle}{
    backgroundcolor=\color{backgroundColour},
    breakatwhitespace=false,
    breaklines=true,
    captionpos=b,
    keepspaces=true,
    numbers=left,
    numbersep=5pt,
    showspaces=false,
    showstringspaces=false,
    showtabs=false,
    tabsize=2,
    numberstyle=\tiny\color{mGray},
    commentstyle=\color{mGreen},
    keywordstyle=\color{magenta},
    stringstyle=\color{mPurple},
    basicstyle=\footnotesize \ttfamily,
    language=Java
}

\lstdefinestyle{CSharpStyle}{
    language=Java,
    basicstyle=\footnotesize \ttfamily,
    backgroundcolor=\color{backgroundColour},
    numbers=left,
    numberstyle=\tiny\color{mGray},
    numbersep=5pt,
    tabsize=2,
    breaklines=true,
    captionpos=b,
    breakatwhitespace=false,
    keepspaces=true,
    showspaces=false,
    showstringspaces=false,
    showtabs=false,
    commentstyle=\color{green},
    keywordstyle=\color{cyan},
    stringstyle=\color{blue},
    identifierstyle=\color{red},
    morekeywords={ abstract, event, struct,
    as, explicit, null, switch,
    base, extern, object, this,
    bool, false, operator, throw,
    break, finally, out, true,
    byte, fixed, override, try,
    case, float, params, typeof,
    catch, for, private, uint,
    char, foreach, protected, ulong,
    checked, goto, public, unchecked,
    class, if, readonly, unsafe,
    const, implicit, ref, ushort,
    continue, in, return, using,
    decimal, int, sbyte, virtual,
    default, interface, sealed, volatile,
    delegate, internal, short, void,
    do, is, sizeof, while,
    double, lock, stackalloc,
    else, long, static, enum, namespace, new, 
    string, await, Assert, var, async, Task}
}


%%%%%%%%%%%%%%%%%%%%%%%%%%%%%%%%%%%%%%%%%%%%%%%%
% Pseudocode
%%%%%%%%%%%%%%%%%%%%%%%%%%%%%%%%%%%%%%%%%%%%%%%%

% These packages allows the possibility to input
% pseudocode - see the file '!PSEDOCODE Help' 
% for simple instruction on how to use
\usepackage{algorithm}
\usepackage{algpseudocode}
\usepackage{program}

\algnewcommand{\algvar}{\texttt}
\algnewcommand{\assign}{\leftarrow}
\algnewcommand{\NIL}{\texttt{NIL}}
\algnewcommand{\NULL}{\texttt{NULL}}


%%%%%%%%%%%%%%%%%%%%%%%%%%%%%%%%%%%%%%%%%%%%%%%%
% TikZ
%%%%%%%%%%%%%%%%%%%%%%%%%%%%%%%%%%%%%%%%%%%%%%%%

% Tikz can be used to generate diagrams and figures
\usepackage{tikz}
\usetikzlibrary{fit, calc, positioning, shapes.geometric, arrows}

\tikzstyle{b} = [rectangle, draw, fill=white!20, node distance=8em, text width=6em, text centered, rounded corners, minimum height=4em, thick]
\tikzstyle{c} = [rectangle, draw, inner sep=0.5cm, dashed]
\tikzstyle{d} = [node distance=0, minimum height=1.5em]
\tikzstyle{g} = [circle, draw, fill=white!5, very thick, minimum size=3em]
\tikzstyle{l} = [draw, -latex',thick, line width=.2em]
\tikzstyle{ll} = [draw, latex'-latex',thick, line width=.2em]
\tikzstyle{p} = [draw,thick]
\tikzstyle{tab} = [midway, text width=10em, align=center]
\tikzstyle{ta} = [tab, above=1.5em]
\tikzstyle{tb} = [tab, below=1.5em]